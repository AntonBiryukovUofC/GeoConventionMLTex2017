\chapter*{Preface}
\addcontentsline{toc}{chapter}{Preface}
%
%\section*{Introduction and motivation}
%
Microseismic techniques have emerged as an important approach for {\em in situ} monitoring of fracture processes, whether for hydrofrac stimulation of tight reservoirs, life-cycle reservoir monitoring for heavy-oil production or mining-related applications. Although relatively new to the oil \& gas industry, microseismic techniques are well established for monitoring deep underground mines, geothermal development, and earthquake monitoring networks, where sophisticated techniques have been honed and developed for decades.
	
The significant increase in interest in microseismicity on all fronts led to the creation of the Microseismic Industry Consortium in January 2010. This applied-research is jointly hosted by the University of Calgary and the University of Alberta. Currently in its third year, the Consortium boasts 31 sponsoring companies representing the oil \& gas, service and mining industries, a wealth of licensed microseismic data to work with, and the world�s first university-based microseismic field acquisition system. Most importantly, this initiative has attracted some extremely talented and resourceful staff and graduate students, many from overseas, whose dedication to the project is truly exemplary.

This initiative has been extremely well supported by our industrial sponsors. This support is deeply appreciated and is of utmost importance for building up capacity and innovation within this emerging field. As our Consortium gains momentum, we will continue to grow in numbers and capitalize on opportunities to leverage this support through federal and provincial funding programs.

This compilation of research results represents the third volume arising from the Microseismic Industry Consortium. Students, researchers and faculty members have been working around-the-clock to bring you this report. Research activities have covered almost the full gamut of geophysical investigations of microseismic activity, from basic processing techniques, interpretation case histories, unconventional examination of microseismic source characteristics and geomechanical analysis to numerical modeling. The sponsor meeting will undoubtedly provide a valuable forum to discuss these results and share new insights and ideas.

We are very pleased to welcome several visiting researchers to our third consortium research meeting. Drs. Amanda Bustin (University of British Columbia), Mike Kendall (University of Bristol, UK) and R. Paul Young (University of Toronto) have accepted our invitations to share their research work with us. We are hoping that these visits will blossom into fruitful collaborations in the months and years to come. We are also very grateful to George Eynon (ERCB), Kellen Foreman (Canadian Associations of Petroleum Producers) and Dan Walker (BCOGC) for agreeing to be part of our panel discussion on Regulatory Practices for Anomalous Induced Seismicity.\\
~


\noindent{\bf David W. Eaton}

\noindent{\bf Mirko van der Baan}

\noindent Directors\\
Microseismic Industry Consortium

\cleardoublepage
%
\chapter{Outline}
%
%\section*{Outline}
%
This progress report contains 29 chapters, approximately divided into 4 categories, namely {\bf review papers} (chapters 2 and 3), {\bf microseismic case histories} (chapters 4--15), {\bf processing} (chapters 16--27) and {\bf geomechanics} (chapters 28--30).

Li and Van der Baan ({\bf Chapter 2}) introduce the concept of rotational seismology, discussing the theory, current acquisition instruments and possible applications.

Van der Baan, Eaton and Dusseault ({\bf Chapter 3}) review current microseismic monitoring developments and research questions in acquisition, processing, interpretation and geomechanics.

Castellanos and Van der Baan ({\bf Chapter 4}) use the double-difference algorithm to obtain high-resolution event locations for a mining dataset.

Eaton and Boroumand ({\bf Chapter 5}) review the energy budget for the Basel hot rock experiment, comparing estimates of injected versus seismically radiated energy.

Eaton, Davidsen and Boroumand ({\bf Chapter 6}) ask the question: Does the magnitude-frequency distribution of microseismic events follow the Gutenberg-Richter relation?

Eaton, Davis, Matthews et al. ({\bf Chapter 7}) provide a progress report for the ongoing Hoadley Flowback Microseismic experiment, undertaken in 2012.

Eaton, Van der Baan, Birkelo et al. ({\bf Chapter 8}) investigate if spectral measurements are sufficient to determine microseismic source parameters.

Eaton, Van der Baan, Tary et al. ({\bf Chapter 9}) examine the low-frequency characteristics of a microseismic experiment in BC, acquired using both broadband surface stations and borehole sensors.

Pike and Eaton ({\bf Chapter 10}) investigate the impact of strong velocity variations on microseismic data processing.

St.Onge and Eaton ({\bf Chapter 11}) show that analysis of Lamb waves recorded in boreholes can reveal pertinent information on the borehole geometry, casing and fluid contents.

St.Onge and Eaton ({\bf Chapter 12}) demonstrate that path effects can strongly influence recorded waveforms.

St.Onge, Eaton and Pidlisecky ({\bf Chapter 13}) suggest a new approach to detect relative stress changes within a monitoring borehole by examining temporal shifts in resonance frequencies due to geophone clamping.

Tary, Van der Baan and Eaton ({\bf Chapter 14}) provide a framework on how to interpret resonance frequencies in continuous microseismic datasets and illustrate the procedure on various case histories.

Vaezi and Van der Baan ({\bf Chapter 15}) analyse the noise levels for a borehole microseismic experiment, concluding that instrument self-noise may surprisingly be a limiting factor in quiet environments. They also suggest a new automated picking method, based on spectral analysis.

Akram and Eaton ({\bf Chapter 16}) assess induced systematic biases in microseismic event locations if seismic velocity anisotropy is ignored.

Akram and Eaton ({\bf Chapter 17}) review a variety of event picking algorithms and discuss their strengths and weaknesses using borehole recordings.

Barthwal and Van der Baan ({\bf Chapter 18}) review the potential of double-difference tomography for analysing time-lapse changes in the velocity field.

Boroumand and Khaniani ({\bf Chapter 19}) apply 2D cross-correlation and a shifted hyperbolic Radon transform for waveform analysis of a double-couple event.

Eaton and Akram ({\bf Chapter 20}) use wavefield reciprocity to improve the accuracy of single event locations using a reversed double-difference algorithm.

Eaton, Maulianda, Hareland et al. ({\bf Chapter 21}) employ the concept of convex hulls to incorporate the effect of location uncertainties into estimations of seismically stimulated volumes.

Feroz and Van der Baan ({\bf Chapter 22}) examine the effect of combining polarization analysis with traveltime minimization for hypocenter determination for horizontal, deviated and vertical boreholes.

Grob and Van der Baan ({\bf Chapter 23}) investigate the effect magnitude-distance detection thresholds have on the shape of microseismic clouds and estimations for Gutenberg-Richter $b$-values and the fractal dimension $D$.

Herrera, Tary and Van der Baan ({\bf Chapter 24}) introduce the synchrosqueezing transform for high-resolution time-frequency analysis.

Jones and Van der Baan ({\bf Chapter 25}) assess the potential of polarization filtering for signal-to-noise enhancement of microseismic recordings.

Malehmir and Van der Baan ({\bf Chapter 26}) review how various microseismic and engineering methods aim at estimating the stimulated reservoir volume.

Tary, Herrera and Van der Baan ({\bf Chapter 27}) introduce two auto-regressive methods for high-resolution time-frequency analysis.

Boroumand and Eaton ({\bf Chapter 28}) use the concept of energy balance to simulate fracture growth during reservoir stimulation.

Chorney et al. ({\bf Chapter 29}) use bonded-particle modeling to simulate rock deformation, demonstrating that not only is the input energy orders of magnitude larger than the energy related to brittle failure but estimates of brittle failure from the seismically radiated energy are 50-100 times too small.

Finally Roche and Van der Baan ({\bf Chapter 30}) describe various numerical strategies for modeling spatial and temporal variations in the dynamic behavior of fracture development, demonstrating that lithological layering can lead to counterintuitive fracture nucleation.
