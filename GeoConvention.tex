% Template for PLoS
% Version 1.0 January 2009
%
% To compile to pdf, run:
% latex plos.template
% bibtex plos.template
% latex plos.template
% latex plos.template
% dvipdf plos.template

\documentclass[letterpaper,11pt]{article}

% amsmath package, useful for mathematical formulas
\usepackage{amsmath}
% amssymb package, useful for mathematical symbols
\usepackage{amssymb}

% graphicx package, useful for including eps and pdf graphics
% include graphics with the command \includegraphics
\usepackage{graphicx}

% cite package, to clean up citations in the main text. Do not remove.
\usepackage{cite}

\usepackage{color} 

% Use doublespacing - comment out for single spacing
%\usepackage{setspace} 
%\doublespacing


% Text layout
\usepackage[top=0.7in, bottom=0.65in, foot=0.7in, left=0.7in, right=0.7in]{geometry}
\usepackage[font={scriptsize}]{caption}
\usepackage[font={scriptsize}]{subcaption}
\usepackage{layout}
\usepackage[scaled]{uarial}
\renewcommand*\familydefault{\sfdefault} 
\usepackage[T1]{fontenc}
%\usepackage{fontspec}
%\setmainfont{Arial}
\usepackage{fancyhdr}
\usepackage{wrapfig}
\setlength{\footskip}{20pt}
\setlength{\parskip}{.2\baselineskip}
\pagestyle{fancy}
\fancyhf{}
\renewcommand{\headrulewidth}{0.0pt}
\renewcommand{\footrulewidth}{0.0pt}
\lfoot{\footnotesize \textit{ GeoConvention 2015: New Horizons} }
\cfoot{}
\rfoot{\footnotesize \thepage}
\lhead{}
\chead{}
\rhead{}
\setlength\parindent{0pt}
\fancypagestyle{sty1}{
\fancyhf{}
\renewcommand{\headrulewidth}{0.0pt}
\renewcommand{\footrulewidth}{0.0pt}
\lfoot{\footnotesize \textit{ GeoConvention 2014: New Horizons} }
\cfoot{}
\rfoot{\footnotesize \thepage}
\lhead{}
\chead{}
\rhead{}
}
% Bold the 'Figure #' in the caption and separate it with a period
% Captions will be left justified
\usepackage[labelfont=bf,labelsep=period,justification=raggedright]{caption}
\usepackage{lipsum}
\usepackage{setspace}
\usepackage{titlesec}
\usepackage{natbib}
\titlespacing\section{0pt}{0pt plus 0pt minus 0pt}{-0pt plus 2pt minus 2pt}
\titlespacing\subsection{0pt}{0pt plus 0pt minus 0pt}{-0pt plus 2pt minus 2pt}
\titlespacing\subsubsection{0pt}{0pt plus 0pt minus 0pt}{-0pt plus 2pt minus 2pt}
% Use the PLoS provided bibtex style
%\bibliographystyle{abbrv}
\frenchspacing
\linespread{0.95}
\begin{document}
%\thispagestyle{sty1}

\begin{center}
\includegraphics[width=2.5in]{header.png} \vspace{-1pt}
\end{center}

% Title must be 150 characters or less
\begin{flushleft}
{\LARGE
\begin{spacing}{0.9}
\textbf{From the lab to real scale: simulation of seismic surveillance in viscoelastic media under varying physical conditions.}
\end{spacing}
}

% Insert Author names, affiliations and corresponding author email.
%\\
\textit{Anton Biryukov, Nicola Tisato, Giovanni Grasselli} \\
\textit{Department of Civil Engineering, University of Toronto}
\end{flushleft}

\section*{Summary}
Estimation of attenuation of the seismic waves in viscoelastic media is important for the quality of geophysical methods relying on processing the recorded waveforms (\cite{marelli_appraisal_2010,madonna_new_2013,nicola_tisato_seismic_2014}).
The purpose of this particular contribution is twofold:(i) propose a methodology to numerically estimate attenuation in the material of interest based on laboratory data; (ii) demonstrate that attenuation can vary with the physical conditions of the medium and how that variation affects signal propagation on the real scale.

\section*{Introduction}
%Subsurface seismics investigations are aimed at obtaining instantaneous subsurface images. However,
There is increasing interest in monitoring changes in the subsurface over time by means of repetitive (i.e. time-lapse) measurements. %what region of interest
Reservoir seismology researchers have spent decades in developing 3D and 4D time-lapse imaging and inversion techniques using active sources and large sensor arrays. 
The extraction of hydrocarbons or injection of water can cause deformation and induce microearthquakes which may interfere with the production process and cause societal concern. Monitoring these effects is increasingly important for optimal production and reservoir management. 

%The technique is currently considered as a reliable option to isolate HLRW from the aquifers and the biosphere (\cite{alexander_mckinley,chapman_2003})
%\quad A number of research projects worldwide investigates the application of non-intrusive seismic monitoring techniques of potential high-level radioactive waste (HLRW) repositories. HLRW are generally enveloped in engineered bentonite barriers and placed in the repository.


\quad Time-lapse imaging involves comparing snap-shot images taken over time, e.g. a section of the reservoir before and after gas injection. Waveform analysis may be applied to locate the areas of anomalous pressure or water saturation (\citet{manukyan_seismic_2012}). For example, research projects worldwide investigates the application of non-intrusive seismic monitoring techniques of potential high-level radioactive waste (HLRW) repositories. HLRW are generally enveloped in bentonite barriers. %unfinished sentence
%Bentonite is a swelling clay that is used as the base material for the barriers as it forms an impermeable (hydraulic conductivity $k\sim10^{-14}$ m/s) seal between the HLRW and the host rock as it saturates (\cite{lajudie_clay-based_1994}). 
The temperature ($T$), pressure, and water content ($W_{c}$) of bentonite is expected to increase dramatically over a few years after completion due to radioactive decay of HLRW, swelling of the clay and water saturation. As a results, the physical state is altered, which may jeopardize the integrity of the barrier (\citet{tisato_laboratory_2013}).%change to 'As a result, the physical state of betonite is altered which may jeorpidize the integrity of the material.'
Seismic monitoring may serve as a non-intrusive tool to track the condition of the barrier.%may be 
Regardless of application, a common requirement has to be satisfied: the difference in the recorded signals must be well discernible and reflect only the changes in the medium properties (\citet{marelli_appraisal_2010}). Therefore, numerical investigation of the procedure in the geometry of interest should be conducted and thoroughly analyzed to aid in design and optimization of the monitoring system.

%Numerical tools provide accurate results if calibrated with laboratory investigation. Thus, numerical modelling of the commonly used experimental techniques is of paramount importance for accurate interpretation of the results. In our previous investigation (\cite{abiryukov_2014}) we proposed a methodology to calibrate 2D numerical tools to yield synthetic traces, accurately reproducing the experimental results of ultrasonic wave propagation in the material of interest.
%\quad Numerical tools provide accurate results if calibrated with laboratory results. Thus, numerical modelling of the common experimental techniques will be helpful for accurate interpretation of the results. In our previous investigation (\cite{abiryukov_2014}) we proposed a methodology to calibrate numerical tools to yield synthetic traces, accurately reproducing the experimental results of wave propagation in the material of interest.
%\quad Signal characteristics depend on the media they travel through. Physical processes affecting the amplitudes have to be considered for better analysis. One of the major factors is called attenuation, and it depends on the physical properties of rock and its fluid content. Attenuation coefficient ($\frac{1}{Q}$) describe the wave amplitude decay while propagating in the rock. Higher $Q$ values correspond to less attenuation; low $Q$ corresponds to greater reduction in seismic amplitude. Shallow formations with large grains and high pore volume are characterized with higher attenuation. This is generally because the relative phase lag in motion of fluids and solids in the rock leads to loss of energy as the wave propagates. This type of energy loss accounts for attenuation and the amount of seismic $Q$ can assist us in monitoring. As high attenution greatly affect the signals' amplitudes and relative phases, it requires consideration during analysis that employs numerical modelling.
\quad Signal characteristics depend on the media they travel through. One of the major factors, affecting the amplitudes, is called attenuation. It depends on the physical properties of rock and its fluid content. Higher values of attenuation coefficient ($\frac{1}{Q}$) correspond to greater reduction in seismic amplitude. As high attenution greatly affect the signals' amplitudes and relative phases, it has to be taken into account for better analysis that employs numerical modelling. 
In this contribution, we numerically estimate the effective attenuation in bentonite as a function of $T$ and $W_{c}$ and show an example of how the variation of $W_{c}$ and $T$ affects the signal propagating through the repository.

%It is also possible to monitor changes in the reservoir continuously. For example, the detection of small earthquakes produced by injection of fluids under high pressure can be used to monitor in situ stress changes in the reservoir. This approach can be used to detect and track changes in the reservoir due to subsurface CO₂ sequestration, gas storage, geothermal energy production, or mining activity. And with seismic interferometry it is now possible to measure minute changes in medium properties as a function of time (Figure 5).

%In the future, these modes of continuous monitoring will conceivably be done in, or near, real time, so that the information about changes in seismicity, in situ stress fields, and reservoir properties will be readily available at the finger tips of operators—thus providing a powerful control tool for reservoir management and optimization.

\section*{Theory and Methods}
\subsection*{Numerical setup}
\citet{tisato_laboratory_2013} measured the longitudinal and transverse ultrasonic wave propagation velocities in bentonite as a function of $T$ and $W_{c}$, reproducing the conditions expected at a HLRW waste repository. Following the iterative procedure described in \citet{abiryukov_2014} we simulated the propagation of ultrasonic waves in a 2D numerical model representing the ultrasonic facility accomodating a bentonite sample (Figure \ref{fig:setup}). The viscoelastic parameters utilized for the materials are reported in Figure \ref{fig:setup} (\citet{lakes_viscoelastic_2009,auerkari_mechanical_1996,boyer_metals_1984}).
\begin{wrapfigure}{r}{0.5\textwidth}
\includegraphics[width=0.5\textwidth]{setup1_refl.png}
 \caption{(a) Schematic diagram of the ultrasonic apparatus employed in the experiments to measure $V_{p}$ and $V_{s}$ and used for iterative attenuation estimation (b) Material parameters used in the numerical simulations.The elastic moduli and density of bentonite vary with the temperature and its water content \citet{tisato_laboratory_2013}; therefore we provide the range of values for $V_{p}$, $V_{s}$, and $\rho$ that were assigned during attenuation estimation. (c) Normalized input signal sent to the emitter}
 \label{fig:setup}
 \end{wrapfigure}
Running simulations with fixed $T$ and $W_{c}$ and varying attenuation for P- and S-waves in bentonite, we compare the numerical and experimental signals in search of the maximum overlap. The value of attenuation, corresponding to the best overlapping simulation, is then regarded as attenution, specific to the fixed values of $T$ and $W_{c}$.
\subsection*{Attenuation modelling} 
Here we define attenuation by the convention proposed by \cite{oconnell_measures_1978}:$\frac{1}{Q}=\frac{M_{i}}{M_{r}}$, where $M_{r}$ and $M_{i}$ are the real and imaginary parts of the complex modulus respectively.

We used a ``generalized standard linear solid'' (GSLS) model (\cite{liu_velocity_1976}) to create the linear viscoelastic rheological model of bentonite samples. 
The model (Figure \ref{fig:attenuation},b) is described by an optimization variable $\tau$ (\cite{blanch_modeling_1995}) and a set of relaxation frequencies $f_{1},f_{2},f_{3}$, representing each Maxwell body. The numerical solver attempts to find optimum parameters of the elements within a set frequency range. Optimum parameters minimize numerically the integrated deviation of the $\frac{1}{Q(f)}$ from the desired attenuation value thus providing the best approximation of a given constant $\frac{1}{Q}$-spectrum :
 \begin{equation}
 \int_{f_{1}}^{f_{2}} [\frac{1}{Q(f,f_{1},f_{2},f_{3},\tau)} - \frac{1}{Q^{'}}]^{2} df \longrightarrow \mathrm{min}
 \label{eq:att_minim}
 \end{equation}
This optimization is performed for the desired $\frac{1}{Q}$ for both P- and S-waves. The resultant attenuation curve has a characteristic shape of multiple peaks in series, located at different relaxation frequencies (\ref{fig:attenuation}).
%\begin{wrapfigure}{r}{0.98\textwidth}
\begin{figure}
\centering
\includegraphics[width=0.8\textwidth]{fig_attenuation.png}
 \caption{(a) $\frac{1}{Q(f)}$ as a function of frequency approximates constant $Q=25$ in 1 kHz -- 1 MHz bandwidth. Note that the signal spectrum remains within the set frequency range at any cross section of the setup (only 4 are shown for clarity). (b) The mechanical representation of linear viscoelastic model (inlet) and corresponding dispersion of S-wave velocity.}
 \label{fig:attenuation}
\end{figure}
%% CORRECTED VELOCITY VERSUS APPARENT VELOCITY
In the experimental framework, \cite{tisato_laboratory_2013} measured apparent S-wave velocities in bentonite samples at 100 kHz ($V_{s,apparent}$ in Figure \ref{fig:attenuation},b). In order to numerically propagate the signal at the correct velocity (i.e. measured apparent velocity), the input value for bentonite $V_{s}$ was corrected according to the dispersion curve and the dominant frequency of the signal. The $V_{s}(f)$ curve was interpolated within the attenuation frequency range with the input value corresponding to the lowest frequency in the range. The same procedure is then applied to the P-wave velocity model.
\section*{Examples and Results}

In this section we report the results: the estimation of attenuation in bentonite as a function of temperature and water content, and typical synthetic seismograms obtained simulating the propagation of elastic waves in an HLRW in which $T$ and $W_{c}$ of bentonite vary.
\subsection*{Attenuation estimation}
We employed the ultrasonic shear wave propagation data recorded for four bentonite samples (10\%, 20\%, 35\% and 52\% water-content) that were progressively heated in four steps of $10^{\circ}~\mathrm{C}$ from $30^{\circ}~\mathrm{C}$ to $70^{\circ}~\mathrm{C}$. Therefore, the values of attenuation for P- and S-waves were estimated on a $4\times5$ grid in the $W_{c}-T$ domain. For the first set of iterations the procedure revealed the location of the series of minima at $Q_{p} \in (9,30)$ and $Q_{s} \in (4,15)$. The second iteration refined the region by further discretization into $10\times10$ grid points. The procedure was then repeated for every node on $T-W_{c}$-grid. The resultant surface plots of quality factors are illustrated in Figure \ref{fig:qpqs}.
\begin{wrapfigure}{r}{0.5\textwidth}
\includegraphics[width=0.5\textwidth]{Qp_Qs.eps}
 \caption{Distribution of $Q_{p}$ and $Q_{s}$ in bentonite as a function of temperature and water content: interpolated $Q_{p}(T,W_{c})$ and $Q_{s}(T,W_{c})$ surface plots}
  \label{fig:qpqs}
\end{wrapfigure}
Our results clearly demonstrate that attenuation in bentonite at fixed temperature $T$ is strongly affected by the water content, with a presence of a distinct minimum at $W_{c}=20\%$. However, both quality factors tend to follow similar trend and show no significant variation with respect to temperature $T$ at fixed values of $W_{c}$. 
\subsection*{Modified GTS framework simulations}
\begin{figure}[htb]
\centering
\includegraphics[width=0.75\textwidth]{fig_testTOFF.png}
 \caption{(a) Schematic diagram of the tonnel cross section B and monitoring system, the viscoelastic parameters of the materials employed (Table) and (b) seismograms, corresponding to modelled bentonite physical conditions (cases).The receivers are indicated as black triangles, the source location is shown in red diamond. The table summarizes the viscoelastic parameters that were assigned to bentonite on a case-by-case basis. Seismic traces are colored based on the physical conditions (modeling case) they refer to.}
  \label{fig:tunnel_off}
\end{figure}

As we evaluated the attenuation properties of bentonite under variable physical conditions, we modelled active seismic monitoring procedure in the modified framework of the GTS experiment. To represent the evolution of bentonite's physical conditions, we chose six cases - combinations of $T$ and $W_{c}$ (Figure \ref{fig:qpqs}). The setup follows the description of the site monitoring system described in \cite{marelli_appraisal_2010}. The seismic energy was released in a form of 3 kHz center frequency Ricker P-wave explosion and recorded by the acquisition system (Figure \ref{fig:tunnel_off}). The geometry of cross section of the tunnel (Figure \ref{fig:tunnel_off}), the relative source and receivers positions, and the results of corresponding numerical simulations are provided in Figure \ref{fig:tunnel_off}, respectively. 
%\begin{wrapfigure}{r}{0.5\textwidth}
\section*{Conclusions}
Similarity in $Q_{p}$ and $Q_{s}$ behavior (i.e. the overall trend and minimum location) suggests that attenuation mechanisms responsible for P- and S-wave dissipation are likely to be similar and exhibit minor variation due to $T$ in the range of $30^{\circ}$-$70^{\circ}$ C. The magnitude of the variation might also indicate that bentonite's intrinsic structure does not experience significant changes due to temperature increase in the said range compared to the changes caused by water content increase. Consequently one may observe minimal discrepancies in the first arrivals between seismograms, recorded at fixed $W_{c}$ with variable $T$ (compare Case III versus Case VI, or Case I versus Case V), than recorded at fixed $T$ with variable $W_{c}$ (compare Case I and Case VI). The coda of the signal, however, shows satisfactory variation among the modelled cases.


The first arrivals recorded by receiver \#9 and \#10 (Figure \ref{fig:tunnel_off}) are $\sim 10$ times lower than those recorded by receivers \#3 and \#4. However, a relatively strong event is observed shortly after, located around $3-4.5$ ms. This arrival corresponds to a wave train passing through the interior of the tunnel and is therefore delayed and attenuated. The proximity of the receiver to the line passing through the source and the center of symmetry results into a higher recorded amplitude and subsequently higher signal-to-noise ratio (compare receivers \#9 and \#10, Figure~\ref{fig:tunnel_off}). Since the difference in traces at the time interval is reflected both by the decay amplitude and the phase shift between peaks, we suggest favoring this informative part of the signal over the first arrivals during the analysis employing the in-barrier receivers. 


% SUGGEST PLACING SOURCES INSIDE THE TUNNEL
The distance between the source and the receiver is an important factor that affects the quality of the postclosure surveillance (\cite{marelli_appraisal_2010,manukyan_seismic_2012}). The higher the fraction of the bentonite component in the total travel path, the more pronounced become the differences in the recorded signals. One of the ways to achieve the maximum fraction of the bentonite component in the total travel path of the signal is to place the seismic source inside the tunnel. This will force the seismic energy to travel a substantial distance in the tunnel. As a result, the first arrivals registered in the host rock, will be affected by bentonite attenuation, facilitating the waveform comparison.

\section*{Acknowledgments}
The authors would like to express gratitude to Simon Harvey of the University of Toronto's Civil Engineering Departments for his suggestions and revisions. 
\newpage
\bibliography{/home/antonb/LaTeX/biblo.bib}
\bibliographystyle{plainnat}
\end{document}

