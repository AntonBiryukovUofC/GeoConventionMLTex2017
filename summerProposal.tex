\documentclass[a4paper,10pt]{article}
\usepackage[utf8]{inputenc}
\usepackage[cm]{fullpage}

% Title Page
\title{Summer research proposal}
\author{Anton Biryukov}

\begin{document}
\maketitle
\section*{Preamble}
Based on the feedback on my contribution to the SSA poster, this summer I would like to proceed with the research on refining the location of induced seismicity shocks using numerical modeling. The poster has not encompassed fully the proposed hypothesis, but rather demonstrated the application without the proof of concept. Therefore, during this summer I would like to appropriately outline the solution, and demonstrate its application and limitations on several examples, with an intent of publishing the results. Below I describe the concepts and potential steps I need to take towards the implementation.
\section*{Concepts}
Earlier I demonstrated the capabilities of the SW4 code of simulating the (visco)elastic wave propagation in a 3D setting. The idea of refining the depth location using 3D modelling has evolved from the inability of 2D code to fully capture the 3D character of the simulation. Although the velocity model of the simulated framework varies in one direction, 3D simulations are still necessary, as the earthquake has a characteristic 3d property - focal mechanism.


The earthquake location refining problem is approached as an optimization problem, i.e. finding the model vector minimizing the pre-defined cost function. The cost function acts as a measure of the similarity between the waveforms. For example, it can be represented by the value of the normalized cross-correlation or the squared algebraic difference between the predicted waveform and data. The latter is less vulnerable to cycle skipping and simultaneously accounts for the time lag and shape similarity. On the other hand, the cross correlation separately describes the time lag and the magnitude of shape similarity, though disregarding the relative scale of the similarities. 


The optimized set of parameters consists of the depth, latitude and longitude of the earthquake. Due to the lack of detailed information on the velocity model, the elastic properties are excluded from the set. As I am concerned with the event location relative to the seismic network, instead of moving the earthquake around and simulating multiple models I will move the seismic network with respect to the earthquake location, with its configuration fixed. Therefore, I can run the simulations only once, but record the ground motions at the whole free surface, and subset the records based on the shape of the network. The earthquake location can be then determined through finding the location of the network, that contains modelled records with the maximum similarity to the real data. The relative shift between the real location of the stations and the predicted, taken with the opposite sign, would indicate the optimal coordinates of the earthquake.

The method proposed in the poster utilizes the P -to-S arrival part of the signal as a sliding window to compute the squared difference, thus disregarding the true P-wave arrival times and the absolute location. While the method aimed at capturing the features of the said part of the signal in the analysis, it failed at obtaining a single and unique solution, and resulted in an ambiguity in the depth parameter. Hopefully, the optimization for the three spatial parameters will allow resolving such an ambiguity.


\section*{Implementation}
The implementation of the modeling framework has been already done by interfacing the SW4 compiled binaries with the front-end GUI written in MATLAB. The signal processing and filtering will be done (as previously) employing ObsPy - Python library for seismological data/signal processing and analysis. Given the compatibility of Python with Matlab, pipelining the model data from the front-end to python scripts seems a simple task.







\end{document}          
